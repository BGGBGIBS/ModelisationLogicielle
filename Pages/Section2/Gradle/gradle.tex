\section{Gradle}\label{sec:gradle}
\subsection{D\'efinition}
\begin{definition}[Gradle]
	Gradle est un système de construction automatisé pour les projets Java et autres, utilisant un fichier de configuration basé sur Groovy pour décrire les tâches de construction et les dépendances.
\end{definition}

\subsection{Cr\'eation d'un projet Gradle via CLI}
\begin{lstlisting}[style=monstyle]
	cd <repertoire ou creer le projet>
	mkdir <nom du projet>
	cd <nom du repertoire cree>
	gradle init
	// tout par defaut sauf application
	gradle build
	gradle run
	gradle test
	
	
\end{lstlisting}

Dans le fichier \emph{build.gradle}, ajouter ces lignes entre les sections \emph{plugins} et \emph{repositories}:

\begin{lstlisting}[style=monstyle]
	apply plugin : 'java';
	apply plugin : 'eclipse';
	apply plugin : 'idea';
\end{lstlisting}

\begin{table}[H]
	\begin{tabular}{|l|l|}
		\toprule
		\textbf{Commande CLI Gradle} & \textbf{Signification} \\ \midrule
		gradle build & Construit les projets et leurs dépendances.\\
		gradle tasks & Affiche toutes les tâches de build disponibles. \\
		gradle test & Exécute les tests unitaires. \\
		gradle run & Exécute une application. \\
		gradle clean & Supprime les fichiers de construction générés. \\
		gradle wrapper & Génère les scripts d'enveloppe Gradle pour exécuter Gradle sans installation. \\
		gradle dependencies & Affiche les dépendances du projet. \\
		gradle help --task & Affiche de l'aide pour une tâche spécifique. \\
		gradle projects & Affiche la hiérarchie de projet. \\ \bottomrule
	\end{tabular}
\end{table}