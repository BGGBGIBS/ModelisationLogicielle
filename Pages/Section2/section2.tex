\chapter{UML}\label{chap:uml}
\begin{definition}[UML — Ubified Modeling Language]
    UML (Unified Modeling Language)→Ensemble de notations pour décrire un système logiciel de manière visuelle. (Langage de modélisation)
\end{definition}
Il permet la spécification, visualisation, simulation, construction, documentation des logiciels.
Il facilite la communication et le travail en équipe.
Il possède différents modèles pour différents points de vue. Utilise une approche orientée objet.
\begin{itemize}
    \item Points Forts
    \begin{itemize}
        \item Langage semi-formel
        \begin{itemize}
            \item Plus de précision
            \item Stable
            \item Encourage l'automatisation
        \end{itemize}
        \item Support de communication efficace
        \begin{itemize}
            \item Cadre l'analyse des besoins
            \item Compréhension des systèmes plus facile
            \item Langage universel (polyvalent/souple)
        \end{itemize}
    \end{itemize}
    \item Points Faibles
    \begin{itemize}
        \item Aprentissage long
        \item Intégration d'UML dans un processus $\implies Difficile$
    \end{itemize}
\end{itemize}
\begin{table}[h!]
	%\centering
	\begin{center}
		\caption{Modélisation des Critères}
		\label{tbl:modélisation}
		\begin{tabular}{c|c|c|c}
			\toprule
			\multirow{2}{*}{Points Forts}\multirow{3}{*}{Langage semi-formel}
			& Minerval		      \\
            & Minerval		      \\
            & Minerval		      \\
			%\midrule
            \cmidrule{1-2} \cmidrule(lr){3-3}\cmidrule(lr){4-4}
			\multirow{1}{*}{Points Faibles}\multirow{3}{*}{Support de communication efficace} 
			& Durée 		      \\
			& Autonomie  	      \\
			& Exigences 	      \\
			\bottomrule
		\end{tabular}
	\end{center}
\end{table}


\section{Diagramme de cas d'utilisation}\label{sec:diagrammecasdutilisation}

\section{Diagramme d'Activités}\label{sec:diagrammedactivite}

\section{Diagramme d'interaction}\label{sec:diagrammedinteraction}

\section{Diagramme de Classes}\label{sec:diagrammedeclasses}

\section{Diagramme de Séquences}\label{sec:diagrammedesequence}

\section{Diagramme d'États Comportementaux}\label{sec:diagrammedetatcomportementaux}
