\subsection{Observer}\label{subsec:observer}
%\begin{definition}[Observer]
%	Le design pattern observer est un patron de conception qui.
%\end{definition}

\begin{table}[H]
\caption{Design pattern Observer}
\label{tbl:design_patterns_observer}
\begin{adjustbox}{max width=\textwidth}
\begin{tabular}{l|p{\textwidth}}
\toprule
\textbf{But} & Permettre à un objet de suivre l'état d'un autre objet et de recevoir une notification en cas de changement d'état.\\
\cmidrule(lr){1-2}
\textbf{Quand} & Lorsque vous voulez que plusieurs objets restent synchronisés et reçoivent une notification en cas de changement d'état de l'un d'entre eux.\\
\cmidrule(lr){1-2}
\textbf{Comment} & Le design pattern Observer définit une relation de type "un-vers-plusieurs" entre un objet observé (Observable) et plusieurs objets observateurs (Observer). L'Observable envoie une notification aux Observer en cas de changement d'état. Les Observer peuvent alors mettre à jour leur état en fonction de celui de l'Observable.\\
\cmidrule(lr){1-2}
\textbf{Avantages} & Permet de maintenir la synchronisation entre plusieurs objets sans avoir à connaître leurs implémentations. Facilite l'ajout ou la suppression d'Observer sans avoir à modifier l'Observable.\\
\cmidrule(lr){1-2}
\textbf{Inconvénients} & Nécessite la création de liens de dépendance entre les objets observés et les objets observateurs. Peut rendre le code plus complexe si utilisé de manière excessive.\\
\cmidrule(lr){1-2}
\multirow{4}{*}{\textbf{\'El\'ements}} & Subject: Aussi appel\'e \'emetteur, c'est l'interface impl\'ement\'e par les \'el\'ements \`a observer.\\
& Observer: Interface qui d\'eclare l'interface de notifications via une m\'ethode update.\\
& Concrete Subject: Impl\'ementation concr\`ete des \'el\'ements \`a observer qui \'emettent des notifications envers les observers. \\
& Concrete Observer: effectue certaines actions en r\'eponses aux notifications des \'el\'ements \'ecout\'es.\\
\cmidrule(lr){1-2}
\textbf{Exemples} & 
\hspace{4mm}
\begin{minipage}[tl]{0.5\textwidth}
\begin{minipage}[t]{1\textwidth}
Mise à jour en temps réel d'un tableau de bord d'un système de gestion de projets. Lorsque des données sont mises à jour dans le système, un observateur est déclenché et met à jour le tableau de bord en conséquence.   
\end{minipage}
\begin{minipage}[b]{1\textwidth}
\begin{lstlisting}[style=monstyle]
import java.util.ArrayList;
import java.util.List;

// Classe Observateur
interface Observateur {
  public void update(int nouvelleDonnee);
}

// Classe Sujet
class Sujet {
  private List<Observateur> observateurs = new ArrayList<Observateur>();
  private int donnee;

  public void ajouterObservateur(Observateur observateur) {
    observateurs.add(observateur);
  }

  public void supprimerObservateur(Observateur observateur) {
    observateurs.remove(observateur);
  }

  public void notifierObservateurs() {
    for (Observateur observateur : observateurs) {
      observateur.update(donnee);
    }
  }

  public void mettreAJourDonnee(int nouvelleDonnee) {
    donnee = nouvelleDonnee;
    notifierObservateurs();
  }
}

// Classe Observateur concr\`ete
class TableauDeBord implements Observateur {
  private Sujet sujet;

  public TableauDeBord(Sujet sujet) {
    this.sujet = sujet;
    sujet.ajouterObservateur(this);
  }

  @Override
  public void update(int nouvelleDonnee) {
    // Mise \`a jour du tableau de bord avec la nouvelle donn\'ee
  }
}

// Classe de test
class Test {
  public static void main(String[] args) {
    Sujet sujet = new Sujet();
    TableauDeBord tableauDeBord = new TableauDeBord(sujet);

    sujet.mettreAJourDonnee(10); // Le tableau de bord sera mis \`a jour avec la nouvelle donn\'ee
  }
}

\end{lstlisting} 
\end{minipage}
\end{minipage}
%\\
%\cmidrule(lr){2-2}
%
\hspace{6mm}
\begin{minipage}[tr]{0.5\textwidth}
\begin{minipage}[t]{1\textwidth}
Notification d'un utilisateur lorsqu'un nouveau message est re\c cu dans une application de messagerie. L'application est configur\'ee pour observer les nouveaux messages et en informer l'utilisateur en envoyant une notification push.  
\end{minipage}
\begin{minipage}[b]{1\textwidth}
\begin{lstlisting}[style=monstyle]
import java.util.ArrayList;
import java.util.List;

// Classe representant un utilisateur de l application de messagerie
class User {
  private String name;
  private List<Message> messages;

  public User(String name) {
    this.name = name;
    this.messages = new ArrayList<>();
  }

  // Methode appelee lorsqu un nouveau message est recu
  public void receiveMessage(Message message) {
    messages.add(message);
    // Envoi de la notification push a l'utilisateur
    sendPushNotification();
  }

  // Methode permettant d envoyer une notification push a l utilisateur
  public void sendPushNotification() {
    System.out.println( Notification envoyee a l utilisateur  + name +  : nouveau message recu ! );
  }
}

// Classe representant un message dans l'application de messagerie
class Message {
  private String content;

  public Message(String content) {    this.content = content; }

  public String getContent() {    return content; }
}

// Classe representant l'application de messagerie
class MessagingApp {
  private List<User> users;
  private List<Message> messages;

  public MessagingApp() {
    this.users = new ArrayList<>();
    this.messages = new ArrayList<>();
  }

  // Methode permettant d'ajouter un utilisateur a l'application
  public void addUser(User user) {    users.add(user); }

  // Methode permettant d'envoyer un message a tous les utilisateurs de l'application
  public void sendMessage(Message message) {
    messages.add(message);
    // Notification de tous les utilisateurs de l'application
    for (User user : users) {
      user.receiveMessage(message);
    }
  }
}

public class Main {
  public static void main(String[] args) {
    // Creation de l'application de messagerie
    MessagingApp messagingApp = new MessagingApp();

    // Creation de deux utilisateurs
    User user1 = new User("Alice");
    User user2 = new User("Bob");

    // Ajout des utilisateurs a l'application de messagerie
    messagingApp.addUser(user1);
    messagingApp.addUser(user2);

    // Envoi d'un message a tous les utilisateurs de l'application
    messagingApp.sendMessage(new Message("Bonjour, comment vas-tu ?"));
  }
}

\end{lstlisting}
\end{minipage}
\end{minipage}
\\
\bottomrule
\end{tabular}
\end{adjustbox}
\end{table}