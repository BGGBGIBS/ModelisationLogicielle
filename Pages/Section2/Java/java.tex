\section{Java}\label{sec:java}
Java est un langage de programmation orienté objet qui a été développé par Sun Microsystems (maintenant propriété de Oracle) dans les années 1990. Il est basé sur le langage C++ et s'est inspiré de nombreux autres langages de programmation pour créer un langage facile à utiliser, simple à apprendre et capable de fonctionner sur toutes sortes de matériel et de systèmes d'exploitation.

Java est un langage de programmation interprété qui utilise le concept de "machine virtuelle" pour exécuter du code sur n'importe quelle plate-forme. Cela signifie que le code Java peut être compilé une fois et exécuté sur n'importe quel ordinateur disposant d'une machine virtuelle Java installée.

Java est également un langage de programmation orienté objet, ce qui signifie qu'il utilise une approche basée sur les objets pour la création de programmes. Les objets sont des éléments de code qui représentent des concepts du monde réel, tels que des personnes, des voitures ou des comptes bancaires. Les objets ont des propriétés (ou attributs) qui décrivent leurs caractéristiques, ainsi que des méthodes qui décrivent ce qu'ils peuvent faire. En utilisant la programmation orientée objet, les développeurs peuvent créer des programmes complexes en regroupant des objets simple en une structure hiérarchique.
Voici quelques-unes de ses principales caractéristiques :
\begin{itemize}
\item Orienté objet : Java utilise le concept de classes et d'objets pour structurer le code et modéliser les données.


\begin{tabular}{|c|c|c|}
\toprule
& \textbf{Orienté objet} & \textbf{Procédural} \\
\midrule
Structure de données & Classes & Structures de données simples \\
Modélisation du monde réel & Objet & Fonctions \\
Organisation du code & Encapsulation & Séparation des fonctions \\
Gestion de la mémoire & Gestion automatique & Allocation/libération manuelle \\
Réutilisabilité du code & Héritage et polymorphisme & Copier-coller de code \\
\bottomrule
\end{tabular}



\item Langage de haut niveau : Java est un langage de haut niveau, ce qui signifie qu'il est proche de la syntaxe de la langue humaine et qu'il est facile à lire et à comprendre.
\begin{definition}[Langage de Haut Niveau]
Un langage de haut niveau est un langage de programmation qui est conçu pour être proche de la langue humaine et facile à lire et à écrire pour les humains. Il utilise des mots et des phrases qui sont proches de la langue courante et qui sont moins proches de la langue machine utilisée par l'ordinateur. Les langages de haut niveau sont généralement conçus pour être utilisés par des développeurs pour écrire du code de manière plus rapide et plus pratique que si elles devaient utiliser des langages de bas niveau.
\end{definition}

\begin{tabular}{|c|c|c|}
\toprule
& \textbf{Langage de haut niveau} & \textbf{Langage de bas niveau} \\
\midrule
\textbf{Nature} & Plus proche du langage humain & Plus proche de la machine \\
\textbf{Abstraction} & Plus élevée & Plus faible \\
\textbf{Complexité} & Moins complexe & Plus complexe \\
\textbf{Exécution} & Nécessite un interpréteur ou un compilateur & Exécuté directement par la machine \\
\textbf{Exemples} & Java, Python, C++ & Assembly, binaire \\
\bottomrule
\end{tabular}


\item Langage indépendant de la plateforme : Java est conçu pour être exécuté sur n'importe quelle plateforme, y compris les ordinateurs de bureau, les serveurs, les appareils mobiles et les appareils embarqués.
\item Sécurité renforcée : Java inclut des caractéristiques de sécurité telles que les contrôles de type de données et de gestion de la mémoire qui aident à protéger contre les erreurs de programmation et les attaques informatiques.
\item Garbage collection automatique : Java inclut un système de ramassage de déchets qui libère automatiquement la mémoire non utilisée, ce qui simplifie la gestion de la mémoire pour le développeur.
\item Programmation multithread : Java prend en charge la programmation multithread, ce qui permet de créer des applications qui effectuent plusieurs tâches en même temps.
\end{itemize}
\subsection{Orienté Objet}
\begin{definition}[Orienté Objet]
L'orienté objet est une approche de la programmation informatique qui vise à modéliser les éléments du monde réel en termes de "objets", chacun ayant des propriétés (ou "attributs") et des comportements (ou "méthodes"). Dans un langage de programmation orienté objet, les programmes sont composés de "classes", qui sont des modèles pour les objets, et chaque objet est une instance d'une classe. Les objets communiquent entre eux en envoyant et en recevant des messages, qui déclenchent l'exécution de méthodes spécifiques. L'orienté objet permet une meilleure modularité et réutilisabilité du code, ainsi qu'une meilleure organisation de celui-ci.
\end{definition}
\begin{definition}[Encapsulation]
L'encapsulation est un concept de programmation orientée objet qui consiste à regrouper dans un même objet les données et les méthodes qui leur sont liées, de façon à les protéger de toute modification ou accès non autorisé. L'encapsulation permet de gérer l'intégrité des données d'un objet et de contrôler l'accès aux différentes parties de l'objet. Elle est réalisée en utilisant les modificateurs d'accès (public, private, protected), qui permettent de définir qui peut accéder aux données et aux méthodes d'un objet.
\end{definition}
\begin{definition}[Polymorphisme]
Le polymorphisme est un concept de programmation orientée objet qui consiste à utiliser un même nom de méthode ou de variable pour plusieurs objets de types différents, en s'assurant que chaque objet réagira de manière appropriée lorsque la méthode ou la variable est utilisée. Cela permet de créer du code plus générique et réutilisable, car il peut être utilisé pour différents types d'objets sans avoir à se préoccuper de leur type spécifique.
\end{definition}
\subsection{Variables}
\subsection{Méthodes}
\subsection{Classes}
