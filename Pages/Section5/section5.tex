\chapter{Mathématique de la gestion}\label{chap:mathdegestion}


\section{Théorie des Graphes}\label{sec:thgraphes}
\subsection{Introduction}
\subsubsection{Historique}
\subsubsection{Modélisation}
\subsection{Définition}
\subsubsection{Niveaux, rangs et circuits}
\subsubsection{Problèmes de colloration}
\subsubsection{Algorithme de Bellman-Kalaba}
\subsubsection{Algorithme de Dijkstra}

\newpage
\section{Niveaux, rangs et circuits}\label{sec:nrc}
\section{L'arbre Partiel Minimum}\label{sec:arbrepartielmini}
\subsection{Arbre}\label{subsec:arbre}
\subsection{Arbre partiel minimum}\label{subsec:arbrepm}
\subsection{Algorithme de Prim}\label{subsec:algodeprim}
\section{Problèmes de coloration}\label{sec:coloration}
\subsection{Coloration des sommets d'un graphe}\label{subsec:sommets}
\subsection{Algorithme de Welsh et Powell}\label{subsec:algowelshpowell}
\section{Algorithme de Bellman-Kalaba}\label{sec:algobellmankalaba}
\section{Algorithme de Dijkstra}\label{sec:algodijkstra}

\newpage
\subsection{Problèmes de théorie des graphes}
\subsubsection{Définitions}
\subsubsection{Terminologie}
\subsubsection{Arbre partiel minimum}
\subsubsection{Coloration}
\subsubsection{Chemins les plus courts et les plus longs}

\newpage
\subsection{Problèmes d'ordonnancement}
\subsubsection{A définition du problème}
\subsubsection{Ordonnancement}
\subsubsection{Contraintes Temporelles}
\subsubsection{Contraintes cumulatives}
\subsubsection{B Méthode du chemin critique}\label{subsubsec:cc}
\subsubsection{Élaboration du graphe}
\subsubsection{Ordonnancement au plus tôt}
\subsubsection{Ordonnancement au plus tard}
\subsubsection{Calcul}
\subsubsection{Chemin Critique}
\subsubsection{Marge}
\subsubsection{Diagramme de Gant}
\subsubsection{Contraintes Cumulatives}\label{subsubsec:cc}
\subsubsection{Courbe de Charge}
\subsubsection{Lissage Manuel}
\subsubsection{Algorithme de MILORD}
\subsubsection{Variantes}
\subsubsection{Méthode des potentiels}
\subsubsection{Méthode PERT - Durée des Tâches}\label{subsubsec:mpert}

\newpage
\subsection{Réseaux de transport}
\subsubsection{Flot Maximum et Coupe minimum} 
\subsubsection{Cas Particuliers et variantes}
\subsection{Problème du voyageur de commerce}
\subsubsection{Prise en compte du coût VS durée de réalisation des tâches}

\newpage
\section{Optimisation - Programmation Linéaire}\label{sec:optpl}
\subsection{Un premier Exemple}
\subsection{D'autres cas réels}
\subsection{Définitions - Notations}
\subsection{Solution Graphique}
\subsection{Résultats Fondamentaux}