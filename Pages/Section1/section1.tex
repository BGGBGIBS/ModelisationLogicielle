\chapter{Introduction}\label{chap:introduction}
\section{Bonnes Pratiques}

\begin{itemize}
    \item Exactitude: dévelopez des modèles syntaxiquement corrects
    \item Précision: évitez l’ambiguité dans les modèles –Concision: évitez les détails inutiles
    \item Complétude: modélisez tous les aspects essentiels –Cohérence: évitez les incohérences dans les modèles
    \item Compréhensibilité: créez des modèles lisibles et compréhensibles
    \item Uniformité: utilisez un style uniforme partout
\end{itemize}
\section{Modélisation}
\begin{definition}[Modèle]
    Un modèle est une représentation simplifiée d’une partie de la réalité dans un but spécifique.
\end{definition}
 Un modèle est toujours une abstraction : \\ Pour réduire la complexité, on ne tient pas compte de certaines caractéristiques ou propriétés de l’objet complexe qu’on modélise, afin de se concentrer sur d’autres
\begin{itemize}
\item Modélisation de la structure statique
\begin{itemize}
    \item Diagrammes de classes
    \item Diagrammes d'objets
\end{itemize}
\item Modélisation de l’architecture
\begin{itemize}
    \item Du logiciel: Les diagrammes de composants
    \item Du matériel: Les diagrammes de déploiement
\end{itemize}
\item Modélisation du comportement dynamique
\begin{itemize}
    \item Modélisation de la fonctionnalité pour l'utilisateur
    \begin{itemize}
        \item les diagrammes de cas d'utilisation
        \item Les scénarios semi-structurés
    \end{itemize}
    \item Modélisation des processus et workflow
    \begin{itemize}
        \item interaction overview diagrams
        \item Les diagrammes d'activités
        \item Les diagrammes de séquences
    \end{itemize}
    \item Les statecharts (diagrammes d'états comportementaux
\end{itemize}

\end{itemize}


\begin{description}
    \item[Modélisation de la structure statique]\\\
    \begin{itemize}
        \item Les diagrammes de classes
        \item Les diagrammes d'objets
        \item Les diagrammes de paquetages
        \item Les \og composite structure diagrams\fg\
        \item Modélisation de l’architecture
        \begin{itemize}
            \item le logiciel: les diagrammes de composants
            \item le matériel: les diagrammes de déploiement
        \end{itemize}
    \end{itemize}

    \item[Modélisation du comportement dynamique]\\\
    \begin{itemize}
        \item Modélisation de l’utilisation
        \begin{itemize}
            \item Les diagrammes de cas d'utilisation
        \end{itemize}
        \item modélisation des activités et états
        \begin{itemize}
            \item Les diagrammes d'états comportementaux (statecharts)
            \item Les diagrammes d'activités
        \end{itemize}
        \item modélisation de l'interaction
        \begin{itemize}
            \item Les diaframmes de séquence
            \item Les diagrammes de communication
            \item Les \og interaction overview diagrams\fg\
            \item Les \og timing diagrams\fg\
        \end{itemize}
    \end{itemize}
\end{description}

\begin{description}
    \item[Vues multiples]
    \item[Diagramme de classes et d'objets] décrit les classes, leurs interrelations, et leurs instances
    \item[Diagramme de séquence et de communication] montre le comportement du système par l’interaction des objets qui le compose.
    \item[Diagramme d’états comportementaux (statecharts)]
    \begin{itemize}
        \item montre comment le système se comporte de façon interne
        \item modélise le comportement discrète piloté par les événements
        \begin{itemize}
            \item Diagramme de composants
        \end{itemize}
        \item Montre comment les différents composants du logiciel sont organisés logiquement
    \end{itemize}
    \item[Diagramme de déploiement]
    \begin{itemize}
        \item montre comment les différents composants (machines) du matériel
(hardware) sont organisés physiquement
    \end{itemize}
    \item[Diagramme de cas d'utilisation]
    \begin{itemize}
        \item Montre les utilisations possibles d'un logiciel
    \end{itemize}
    \item[Diagramme d'activités]modélise le comportement d’un système ou processus basé sur les flux de contrôle
\end{description}
