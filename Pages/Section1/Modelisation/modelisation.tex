\section{Modélisation}\label{sec:diagramme_modelisation}
\begin{definition}[Modèle]
    Un modèle est une représentation simplifiée d’une partie de la réalité dans un but spécifique.
\end{definition}
Un modèle est toujours une abstraction: pour réduire la complexité, on ne tient pas compte de certaines caractéristiques ou propriétés de l’objet complexe qu’on modélise, afin de se concentrer sur d’autres.

\begin{itemize}
    \item Exactitude: dévelopez des modèles syntaxiquement corrects
    \item Précision: évitez l’ambiguité dans les modèles 
    \item Concision: évitez les détails inutiles
    \item Complétude: modélisez tous les aspects essentiels 
    \item Cohérence: évitez les incohérences dans les modèles
    \item Compréhensibilité: créez des modèles lisibles et compréhensibles
    \item Uniformité: utilisez un style uniforme partout
\end{itemize}


\begin{definition}[UML — Unified Modeling Language]
    UML (Unified Modeling Language) est un ensemble de notations pour décrire un système logiciel de manière visuelle en termes de modèles (Langage de modélisation).
\end{definition}


\begin{table}[H]
	\centering
		\caption{Langage UML}
		\label{tbl:modélisation}
		\begin{adjustbox}{max width=\textwidth}
		\begin{tabular}{l|l|l}
			\toprule
			\multirow{6}{*}{Points Forts}&\multirow{3}{*}{Langage \og semi-formel\fg et standardisé}		&  Gain de précision\\
			\cmidrule(lr){3-3}
            & & Gage de stabilité\\
            \cmidrule(lr){3-3}
            & & Encourage l’automatisation via l’utilisation d’outils dédiés\\
            \cmidrule(lr){2-2}\cmidrule(lr){3-3}
            &\multirow{3}{*}{Support de communication efficace}
			&  Cadre l'analyse des besoins\\
			\cmidrule(lr){3-3}
            & & Facilite la compréhension de systèmes abstrait et complexes\\
            \cmidrule(lr){3-3}
            & & Langage universel car polyvalent et souple\\
			%\midrule
            \cmidrule{1-2} \cmidrule(lr){3-3}
			\multirow{4}{*}{Points Faibles}& \multirow{2}{*}{Nécessité d'apprentissage et de période d’adaptation} 
			& UML contient beaucoup (trop?) de diagrammes ...  \\
			\cmidrule(lr){3-3}
			&& On peut commencer avec un sous-ensemble \\
			\cmidrule(lr){2-2}\cmidrule(lr){3-3}
			&\multirow{2}{*}{Absence du processus, clé de la réussite d’un projet}
			& Intégration d'UML dans un processus $\implies Difficile$\\
			\cmidrule(lr){3-3}
			&& Améliorer un processus est une tâche longue et complexe\\
			\bottomrule
		\end{tabular}
		\end{adjustbox}
\end{table}

\begin{table}[H]
\centering
%\begin{center}
\caption{Catégories et sous-catégories de diagrammes UML}
\label{tbl:uml}
\begin{adjustbox}{max width=\textwidth}
\begin{tabular}{p{7em}|p{6em}|p{10em}|p{40em}}
\toprule
\multicolumn{2}{c}{Modélisation} & Type de diagramme & Description \\
\midrule
\multirow{6}{7em}{Structure Statique} 
& \multirow{4}{*}{Objet}& \nameref{sec:diagramme_classe} 
& Décrit les classes, leurs interrelations, et leurs instances. \\
\cmidrule(lr){3-4}
&& Diagramme d'objets & Décrit les objets et leurs relations avec les classes. \\
\cmidrule(lr){3-4}
&& Diagramme de paquetages & Montre la répartition des éléments dans des groupes logiques. \\
\cmidrule(lr){3-4}
&& Diagramme de structure composite & Représente la structure interne de composants de logiciel. \\
\cmidrule(lr){2-4}
&\multirow{2}{*}{Architecture} & Diagramme de composants (logiciel) & Montre comment les différents composants du logiciel sont organisés logiquement. \\
\cmidrule(lr){3-4}
&& Diagramme de déploiement (matériel) & Montre comment les différents composants du matériel (hardware) sont organisés physiquement. \\
\cmidrule(lr){1-4}
\multirow{7}{7em}{Comportement Dynamique} 
& \multirow{1}{*}{Utilisation} & \nameref{sec:diagramme_casutilisation} 
& Montre comment un système interagit avec ses utilisateurs.\\
\cmidrule(lr){2-4}
&\multirow{2}{6em}{Activités et États} 
& \nameref{sec:diagramme_etatscomportementaux} & Montre comment le système se comporte de façon interne, modélise le comportement discret piloté par les évènements.\\
\cmidrule(lr){3-4}
&& \nameref{sec:diagramme_activite} & Modélise le comportement d’un système ou processus basé sur les flux de contrôle.\\
\cmidrule(lr){2-4}
&\multirow{4}{*}{Interaction} 
& \nameref{sec:diagramme_sequence} & Montre le comportement du système par l’interaction des objets qui le compose.\\
\cmidrule(lr){3-4}
&& Diagramme de communication & Montre comment les objets communiquent entre eux.\\
\cmidrule(lr){3-4}
&& \nameref{sec:diagramme_interaction} & Montre comment les processus et interactions sont organisés dans le système.\\
\cmidrule(lr){3-4}
&& Timing diagram & Montre comment les interactions et les changements d'états sont liés dans le temps.\\
\bottomrule
\end{tabular}
\end{adjustbox}
%\end{center}
\end{table}
