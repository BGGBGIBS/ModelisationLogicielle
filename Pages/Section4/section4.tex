\chapter{Analyse d'Algorithmes}\label{chap:algorithmes}
\section{Comparaison Asymptotique de Fonctions}\label{sec:caf}
\subsection{Notation}\label{subsec:notation}
\subsection{Autres définitions}\label{subsec:autresdef}
\subsection{Interprétation}\label{subsec:interpretation}
\subsection{Extension à des fonctions partielles}\label{subsec:extfctspart}
\subsubsection{Propriétés}\label{subsec:propriétés}
\subsection{Relation d'équivalence}\label{subsec:relequival}
\subsection{Fonctions Polynomiales}\label{subsec:fctspoly}
\subsection{Fonctions Exponentielles}\label{subsec:fctsexpo}
\subsection{Fonctions Logarithmiques}\label{subsec:fctslog}
\subsection{Comparaison}\label{subsec:compa}
\subsection{Complexité d'un algorithme}\label{subsec:complexitéalgo}
\section{Complexité}\label{sec:complexité}
\subsection{Principe de comptage}\label{subsec:comptage}

\section{Problèmes de coloration}\label{sec:coloration}
\subsection{Coloration des sommets d'un graphe}\label{subsec:sommets}
\subsection{Algorithme de Welsh et Powell}\label{subsec:algowelshpowell}
\section{Algorithme de Bellman-Kalaba}\label{sec:algobellmankalaba}
\section{Algorithme de Dijkstra}\label{sec:algodijkstra}

\begin{table}[h!]
	\begin{center}
		\caption{Comparaison Phi de 3 actions issues de 3 catégories différentes.}
		\label{tbl:phidocaltpro}
		\begin{tabular}{|l|S[table-format=1.4]|S[table-format=1.4]|S[table-format=1.4]|}
			\toprule
			\textbf{Action} & \textbf{$\phi—$} & \textbf{$\phi$} & \textbf{$\phi+$}\\
			%& $\phi+$ & $\phi$ & $\phi-$ \\
			\midrule
			Doctorat & 0,4026 & 0,0822 & 0,4848\\
			Master Alternance & 0,2963 & 0,0852 & 0,3815\\
			Bachelier Professionnalisant & 0,2836 & 0,0800 & 0,3636\\
			\bottomrule
		\end{tabular}
	\end{center}
\end{table}
